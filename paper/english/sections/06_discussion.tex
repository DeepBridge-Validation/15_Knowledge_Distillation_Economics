\section{Discussion}

\subsection{Main Findings}

\subsubsection{Acceptable Trade-off}

Results demonstrate favorable trade-off between accuracy and interpretability:

\begin{itemize}
    \item \textbf{Minimal accuracy loss}: 2-5\% vs. complex teacher models
    \item \textbf{Substantial interpretability gain}: +26 points vs. standard KD
    \item \textbf{Coefficient stability}: CV $< 0.15$ allows rigorous statistical inference
    \item \textbf{Economic conformity}: 95\%+ of constraints preserved
\end{itemize}

\textbf{Implication}: For applications where interpretability is essential (policy analysis, regulation), sacrificing 2-5\% in accuracy is justifiable.

\subsubsection{Superiority vs. Traditional Models}

Economic KD dominates traditional approaches:

\begin{itemize}
    \item \textbf{vs. Linear/Logistic}: +8-12\% AUC, maintaining interpretability
    \item \textbf{vs. GAM Vanilla}: +3-4\% AUC, same interpretability
    \item \textbf{vs. XAI (SHAP/LIME)}~\cite{lundberg2020local,ribeiro2016should}: Intrinsic interpretability (not post-hoc)
\end{itemize}

\textbf{Conclusion}: Framework fills the gap between limited traditional models and opaque ML.

\subsubsection{Stability Validation}

Bootstrap analysis demonstrates coefficients sufficiently stable for:

\begin{enumerate}
    \item \textbf{Statistical inference}: Valid confidence intervals
    \item \textbf{Policy analysis}: Non-volatile conclusions under sampling
    \item \textbf{Reproducibility}: Consistent results across CV folds
\end{enumerate}

\textbf{Contrast}: Standard KD produces coefficients with CV 0.20+ (unstable for inference).

\subsection{Practical Implications}

\subsubsection{For Financial Industry}

\textbf{Regulatory Compliance}:
\begin{itemize}
    \item Basel III / IFRS 9 require interpretable models with statistical foundation
    \item Economic KD produces auditable GAM coefficients for regulators
    \item Stability allows documentation of confidence intervals
\end{itemize}

\textbf{Competitive Advantage}:
\begin{itemize}
    \item Banks can use complex ensembles internally (teacher)
    \item Distill to interpretable GAM for regulatory submission
    \item Minimal accuracy loss (2-3\%) vs. direct use of linear
\end{itemize}

\subsubsection{For Public Policy Makers}

\textbf{Impact Analysis}:
\begin{itemize}
    \item Stable marginal effects allow projection of policy impact
    \item Example: 10\% increase in minimum wage $\rightarrow$ +X\% employment probability
    \item Confidence intervals quantify uncertainty
\end{itemize}

\textbf{Break Detection}:
\begin{itemize}
    \item Automatic identification of structural changes (e.g., 2008 crisis)
    \item Allows adaptation of policies to new economic regimes
\end{itemize}

\subsubsection{For Academic Research}

\textbf{ML-Econometrics Integration}:
\begin{itemize}
    \item Bridge between ML predictive power and econometric rigor
    \item Stable coefficients allow hypothesis testing
    \item Compatible with causal inference (IV, diff-in-diff)
\end{itemize}

\subsection{Limitations}

\subsubsection{1. Constraint Specification}

\textbf{Limitation}: Framework requires economist to specify constraints a priori.

\textbf{Implications}:
\begin{itemize}
    \item Incorrect constraints can degrade accuracy without interpretative gain
    \item Economists may disagree about appropriate constraints
    \item Features without clear theory (e.g., ZIP code) are difficult to constrain
\end{itemize}

\textbf{Mitigation}:
\begin{itemize}
    \item Provide constraints based on established economic literature
    \item Allow relaxation of constraints if violation is systematic
    \item Empirical validation: If unconstrained model violates theory, constraint is justified
\end{itemize}

\subsubsection{2. Interaction Complexity}

\textbf{Limitation}: GAMs are additive---do not capture higher-order interactions.

\textbf{Example}: Effect of education may depend on age (interaction)
\begin{equation}
\text{Effect}(\text{education} | \text{age}) \neq \text{constant}
\end{equation}

\textbf{Future Extension}:
\begin{itemize}
    \item GA$^2$Ms (Generalized Additive Models with explicit interactions)
    \item Constraints on specific interaction terms
\end{itemize}

\subsubsection{3. Causality vs. Correlation}

\textbf{Limitation}: Distillation preserves teacher correlations, not necessarily causal relationships.

\textbf{Example}: Teacher may use proxy variables (e.g., ZIP code $\rightarrow$ race)

\textbf{Implication}:
\begin{itemize}
    \item Coefficients are predictive, but not necessarily causal
    \item Policy analysis requires additional validation (e.g., instrumental variables)
\end{itemize}

\textbf{Future Work}:
\begin{itemize}
    \item Integrate causal discovery in distillation process
    \item Ensure constraints reflect causal structures, not just correlations
\end{itemize}

\subsubsection{4. Scalability}

\textbf{Limitation}: Bootstrap with 1,000+ samples is computationally expensive.

\textbf{Execution Time} (credit dataset, 250k samples):
\begin{itemize}
    \item Teacher training (XGBoost): 15 min
    \item Single distillation run: 8 min
    \item Bootstrap 1,000 runs: $\sim$130 hours (parallel: 8 hours on 16 cores)
\end{itemize}

\textbf{Optimizations}:
\begin{itemize}
    \item Parallelization via joblib/Dask
    \item Bootstrap on subsamples (e.g., 50\% of data)
    \item Analytical variance approximations (future)
\end{itemize}

\subsubsection{5. Generality of Constraints}

\textbf{Limitation}: Constraints may be context/period-specific.

\textbf{Example}: Age $\rightarrow$ default relationship may change in economic crises.

\textbf{Approach}:
\begin{itemize}
    \item Structural break detection identifies changes
    \item Re-specify constraints by period if necessary
    \item ``Soft'' constraints (penalization) vs. ``hard'' (absolute constraint)
\end{itemize}

\subsection{Theoretical Implications}

\subsubsection{Knowledge Distillation as Economic Regularization}

Framework can be viewed as:

\begin{equation}
\min_{\theta} \underbrace{\mathcal{L}_{\text{fit}}(\theta)}_{\text{Accuracy}} + \lambda \underbrace{\mathcal{R}_{\text{econ}}(\theta)}_{\text{Economic Regularization}}
\end{equation}

where $\mathcal{R}_{\text{econ}}$ penalizes violations of economic theory.

\textbf{Interpretation}: Economic constraints act as Bayesian prior informed by decades of research.

\subsubsection{Prediction-Inference Reconciliation}

Mullainathan \& Spiess~\cite{mullainathan2017machine} argue that ML focuses on prediction, econometrics on inference.

\textbf{Our Contribution}: Economic KD reconciles both:
\begin{itemize}
    \item \textbf{Prediction}: Distillation from complex teacher provides accuracy
    \item \textbf{Inference}: GAM student + bootstrap provide stable coefficients with CIs
\end{itemize}

\subsubsection{Interpretability as Constraint Optimization}

We define economic interpretability as an optimization problem:

\begin{align}
\max \quad & \text{Accuracy}(M) \\
\text{s.t.} \quad & \text{Compliance}(M, C) \geq \tau_{\text{compliance}} \\
& \text{Stability}(M) \geq \tau_{\text{stability}} \\
& M \in \{\text{GAM, Linear}\}
\end{align}

Framework approximately solves this multi-objective problem.

\subsection{Comparison with Alternative Approaches}

\subsubsection{vs. Direct Constrained Optimization}

\textbf{Alternative}: Train GAM directly with economic constraints (without distillation).

\textbf{Our Results}: Economic KD surpasses direct constrained GAM by +3-4\% AUC.

\textbf{Explanation}: Complex teacher captures patterns that direct GAM cannot, but distillation transfers knowledge while preserving constraints.

\subsubsection{vs. Post-hoc Calibration}

\textbf{Alternative}: Train complex model, adjust coefficients post-hoc for conformity.

\textbf{Problem}:
\begin{itemize}
    \item Manually adjusted coefficients lack statistical foundation
    \item Calibration may introduce inconsistencies
    \item Does not guarantee stability
\end{itemize}

\textbf{Economic KD Advantage}: Constraints integrated into training, not imposed post-hoc.

\subsubsection{vs. Hybrid Ensembles}

\textbf{Alternative}: Ensemble of complex model + interpretable model.

\textbf{Example}: $\text{Prediction} = 0.7 \times \text{XGBoost} + 0.3 \times \text{GAM}$

\textbf{Problem}:
\begin{itemize}
    \item Interpretability compromised (opaque ensemble)
    \item GAM coefficients do not reflect final prediction
\end{itemize}

\textbf{Economic KD Advantage}: Single student model, fully interpretable.

\subsection{Future Directions}

\subsubsection{Methodological Extensions}

\begin{enumerate}
    \item \textbf{Causal Distillation}: Ensure preservation of causal structures (via causal graphs)
    \item \textbf{Multi-Task Distillation}: Distill for multiple economic objectives simultaneously
    \item \textbf{Adaptive Constraints}: Learn optimal constraints from data (not specify a priori)
    \item \textbf{Intersectionality}: Constraints on subgroups (e.g., education effect varies by gender/race)
\end{enumerate}

\subsubsection{New Domains}

\begin{itemize}
    \item \textbf{Macroeconomics}: Forecasting indicators (GDP, inflation) with interpretability
    \item \textbf{Environmental Economics}: Carbon pricing models with sustainability constraints
    \item \textbf{Behavioral Economics}: Decision models preserving bounded rationality assumptions
\end{itemize}

\subsubsection{Integration with Existing Tools}

\begin{itemize}
    \item \textbf{EconML}: Combine causal inference with economic distillation
    \item \textbf{DoWhy}: Integrate causal reasoning in distillation process
    \item \textbf{Fairlearn}: Add fairness constraints to economic constraints
\end{itemize}
